% DOCUMENT
\documentclass[a4paper]{article}

% PACKAGES
\usepackage{titlesec}
\usepackage[english]{babel}
\usepackage{blindtext}
\usepackage{graphicx} 
\usepackage{enumitem}
\usepackage{amsmath}
\usepackage{mathptmx}
\usepackage{layout}
\usepackage{geometry}
\usepackage{float}

% SETTINGS
% Title Format
\titleformat*{\section}{\large\MakeUppercase}
\titleformat*{\subsection}{\large}
\titleformat*{\subsubsection}{\large}
% Remove paragraph indentation
\setlength\parindent{0pt}
% Set margins
\geometry{margin=1in}

% TITLE
\title{\Huge{2D Filters}}
\author{Faizudeen Kajogbola} 
\date{} %BLACK DATE TO  OMIT DATE FROM TITLE

% DOCUMENT
\begin{document}%\layout
% DISPLAY TITLE
\maketitle

\setlength{\headsep}{5pt}
% \setlength{\voffset}{-0.75in}   % REMOVE EXCESSIVE SPACE AT TOP OF PAGE


\section{Image Correlation and Template Matching}

Image Correlation is an assessment of how alike a pair of images look, 
it is essentially a measure of similarity between two (or more) images. 
In geometric sense, if we consider 2 images, each represented as a matrix of integers ranging from 0 to 255. 
We can stretch out such matrices into two vectors "a" and "b". 
Then the correlation of the 2 images can be expressed as the dot product of the 2 vectors, 
which essentially tells us how much vector "a" is pointing it the direction of vector "b".  \\

Normally, some information on boundary pixels is lost by performing correlation. 
If we take an example of a 5x5 matrix representation of an image $I$ to be correlated with a 3x3 image template $T$. 
If we place the 3x3 template atop the top left corner of the image, the first possible pixel to map would be $I(1,1)$
$-$ assuming zero indexing$-$, effectively making it impossible to map pixels located across row(0) and along column(0). 
Examining the situation at the bottom right corner shows the similar results for the pixels located across row(4) and along column(4). 
In general terms, correlating an image with a template of size NxM means it would not be possible to map pixels located across $\frac{n-1}{2}$ rows 
from the top and bottom borders and across $\frac{m-1}{2}$ columns from on sides. 
In total $n-1$ rows and $m-1$ columns would not be mapped, this is why a correlation operation between an image of size $(WxH)$ 
and a template of size $(nxm)$ results in an image of size $(N-n+1, M-m+1)$. 
A way to fix this would be to pad the image matrix on each side with $\frac{m-1}{2}$ column vectors of zeros, and $\frac{n-1}{2}$ row vectors of 
zeros at the top and bottom borders. \\


\section{Image Smoothing and Image Filtering}

Image Filtering is the process of modifying an image to give some desired characteristics, 
this is generally performed by designing a filter that performs the required modification and applying the filter to the image.  \\

In local filtering$-$ e.g. Histogram Equalization$-$, every pixel is transformed into some new pixel value based on its current value. 
However, in area filtering techniques such as average smootinh filtering, a pixel together with its surrounding pixels are transformed to obtain a new pixel value to which the pixel maps. \\

Linear filters transform a pixel value into a new pixel value using linear functions, 
while non-linear filters perform the transformation with non-linear functions. 
Gaussian filtering and median filtering are examples of linear and non-linear filters respectively.  \\

High-pass filters attenuate (or block) low frequency signals while amplifying/leaving (or passing) high frequency ones, they are often used in edge detection. 
Low-pass filters on the other hand, amplify/leave (pass) low frequency signals while attenuating (blocking) high frequency ones, they can be used to blur an image. \\


\section{Image Noises and Image Sharpening}

\subsection{Image Noise}
Noise in an image is any extreaneous pixel that becomes a part of the image data. 
\begin{itemize}
        \item \textbf{Salt and Pepper Noise:} These are white (255) or black (0) pixels that corrupt image data at random locations. 
        Mostly result from manufacturing defect often called "broken pixels" and can often be corrected by interpolation.  
        \item \textbf{Impulse Noise:} These are randomly placed white pixels in image data. Mostly result from "broken pixels".   
        \item \textbf{Gaussian Noise:} Pixels with varying intensities which are drawn from a Gaussian normal distribution. 
        This is the most common type of noise.  
\end{itemize}


\subsection{Image Sharpening}

Image Sharpening is the process of enhancing the details of an image. 


\section{Fourier Transform and its Application in Image Processing}

It is useful to transform an image from spacial domain to frequency domain because mathematical operations such as convolution can be 
performed more efficiently in the frequency domain. 
It can be learned from the convolution theorem that convolution in the spacial domain is equivalent to simple multiplication in the 
frequency domain. \\ 

When we transform an image into the frequency domain, noise, strong edges, and regions with large pixel intensity variations are 
captured in the high frequencies while the low frequencies represent regions in the image that have smooth gradients. \\

The Laplacian is a high pass filter because if we examine its Fourier Transform, 
we will notice a dark realm around the low frequency region. 
This implies that the Laplacian blocks low frequency components. \\




\pagebreak

\bibliographystyle{plain}
\bibliography{references} 

\end{document}